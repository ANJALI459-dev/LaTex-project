\documentclass[12pt, a4paper]{article}
\usepackage{graphicx} % Required for inserting image
\usepackage{amsmath}
\title{\boldsymbol{ASSIGNMENT}\\ 
{SEC-3 : \LaTeX  Typesetting for Beginners}}
\author{ANJALI KMV}
\date{November 2024}
\setcounter{tocdepth}{0}
\usepackage{blindtext}
\usepackage{fancyhdr}
\usepackage{url}
\fancyhead[LO]{Assignment: ANJALI}
\fancyhead[RO]{SEC-3 \LATEX Typesetting for Beginners}
\fancyfoot[CO]{\spadesuit\boxed{\thepage}\spadesuit}
\pagestyle{fancy}
\begin{document}

\maketitle 

\tableofcontents

\section{Introduction}
\section{About me}
\section{The Mathematics I studied}
\subsection{Exposure to high school algebra}
\subsection{Greek: the new alphabet}
\subsection{Triangles and the trigonometry}
\subsection{ Calculus and calculations}
\subsection{Matrices: the ultimate savior}
\section{\LaTeX  typesetting}
\subsection{My wedding invitation}
\newpage
\section*{1   Introduction} 
    This is the assignment we’re asked to submit as part of the course ‘SEC-3 LATEX Typesetting for Beginners’1.
I did a lot of hard work to complete this. Few of the books and references that helped are the following.
\begin{itemize}
    \item Well-written books
    \begin{itemize}
        \item \LaTeX Beginner's Guide' written by Stefan Kottwitz[2].
        \item \emph{LaTeX} for Beginner's written by K.B.M. Nambudiripad [3].
    \end{itemize}
    \end{itemize}
 \begin{itemize}
     \item Online resources 
     \begin{itemize}
     \item An interactive guide available on the website
    \Url{https://www.overleaf.com/learn/latex/learn_LaTeX_in_30_minutes}.   \\          
     \item The tutorial available on 
     \Url{ https://latex-tutorial.com/tutorials/}.
    \end{itemize}    
   \end{itemize}
 I discussed these things with my friends and fellow students also. They were also very helpful.
\begin{enumerate}
    \item those  who helped throughout the duration of the course
    \begin{enumerate}
        \item khushi
        \item kirti
    \end{enumerate}
        \item  Those who helped in preparing this assignment and the presentation.
        \begin{enumerate}
            \item khushi
            \item kirti
        \end{enumerate}
\end{enumerate} 
 It was a great learning experience. Thanks to Department of Mathematics, Keshav Mahavidyalaya for giving this opportunity. \\
$\boldsymbol{ANJALI} $
\footnote{Offered by the mathematics Department, Keshav Mahavidyalaya}
 \newpage
 \section*{2 About me } 
 My name is  $\boldsymbol{ANJALI} $. I was born in 
 $ \boldsymbol{2003}$,at my family as a second child 
\begin{figure}
    \centering
    \includegraphics[width=0.5\linewidth]{figure/WhatsApp Image 2024-11-21 at 19.15.00 (4).jpeg}
    \caption{That's me}
    \label{fig:enter-label}
\end{figure}
 I completed my school education from  $\boldsymbol{S.K.V PRASHANT VIHAR} $
 I’m currently enrolled in $\boldsymbol{PHYSICAL SCIENCE WITH COMPUTER SCIENCE} $  at Keshav Mahavidyalaya which is a constituent college of university of Delhi. I’m in the 3rd semester and I’m currently studying the following courses. \\      
\begin{tabular}{|c|c|c|c|}
\hline
     \textbf{SI. NO.} & \textbf{Course Type} & \textbf{Course Name} & \textbf{Teacher's Name} \\
\hline
     1 & $DSC$ & Computer system architecture & $Anand$ \\
\hline
     2 & $DSC$ & Differential Equation & $Dhanpal Singh$ \\
\hline
     3 & $DSC$ & Heat and Thermodynamics & $kiran$  \\
\hline
     4 & $DSE$ & Python Programming for data Handling & $Namita Rani$  \\
\hline
      5 & SEC & Latex Typesetting for Beginners &  $Richie Aggarwal$\\
\hline
     6 & VAC & $Vedic Mathmatics-3$  & $Deepak$  \\
\hline
     7 & AEC & Hindi - B & $Keshav Dahiya$ \\
\hline  
\end{tabular}
\section*{3  The Mathematics I studied at my school}     
     Since my childhood, I was very much interested in Mathematics.
\subsection*{ 3.1  Exposure to high school algebra}         
 In my high school I was introduced to basic algebra. We studied identities of the form

   $$  (a + b)^2 = a^2 + 2ab + b^2 $$ 
    $$     (a - b)^2 = a^2 - 2ab + b^2  $$
  $$  a^2 - b^2 = ( a + b )( a - b ) $$
   $$  ( a + b + c ) =  a^2  + b^2 + c^2 + 2ab + 2bc + 2ac $$
The expression given in Equation (1) gives the following. 
\begin{equation}
    X =\frac{-b\pm\sqrt{b^2-4ac}}{2a}
\end{equation}
\boldsymbol{Theorem 3.1.} 
The quadratic equation \(a^2 + bx + c = 0\), where \(a,b,c \in R\) and \(a \neq 0\), has
\begin{enumerate}
    \item two solutions if the discriminant \(b^2 - 4ac \)
    \begin{itemize}
        \item two real solutions if \(b^2 - 4ac > 0\)  , and
        \item  two complex (imaginary) solutions if \(b^2 - 4ac < 0\)  .
    \end{itemize}
    \begin{enumerate}
        \item a unique real solution if \(b^2 - 4ac = 0\) . \\
         Proof. It is an easy exercise.
    \end{enumerate}
\end{enumerate}        
\subsection*{ 3.2  Greek: the new alphabet}
 In Mathematics, we started seeing Greek letters like 
 \(\alpha, \beta , \gamma , \delta , \epsilon , \theta, \rho , \phi, \psi, \xi , \omega , \kappa, \chi, \Phi, \Psi, \Omega\)   ,etc more often (than in our physics class)
 \subsection*{3.3  Triangle and the calculations}
  When I was in Class IX, we were taught trigonometry. The trigonometric functions \(sin\theta, cos\theta, tan\theta\) etc. were mysterious. As I remember
 
$$   sin\theta = \frac{ side adjacent to \theta)}{hypotenuse} $$
  
   in a right angled triangle.
\subsection*{3.4  Calculus and the calculations}
 Even though I enjoyed doing calculations, it was difficult to remember the derivatives and integrals of a lot of functions, even just the trigonometric functions mentioned in Section 3.3. We were asked to learn all these formulas and be ready to tell them even if we were asked while sleeping! A few from that long list of formulas  are given in the following table.
 
\begin{tabular}{|c|c|c|c|c|}
\hline
     \# & \(f(x)\) & \(\frac{d}{dx} f(x)\)   & \(\int f(x)dx*\) & Comments \\
\hline
     1 & x & 1 & \(\frac{x^2}{2}\) & Easy  \\
\hline
     2 & $x^2$ & \(2x\) & \(\frac{x^3}{3}\) & Fine  \\
\hline
     3 & \(sin x\) & \(cos x\) & \(-cosx\) & Comparatively easy \\
\hline
     4 & \(cos x\) & \(-sin x\) & \(sinx\) & Confusing with the previous \\
\hline
      5 & \(tan x\) & \(sec^2 x\) & \(-log|cosx|\) & Difficult\\
\hline
     6 & \(tan^-1x\) & $\frac{1}{1 + x^2}$  & \(xtan^-1x - \frac{1}{2}log(1 + x^2) + c\) & Impossible to remember! \\
\hline
\end{tabular}
  In calculus,we even dealt with the following kind of functions. \\
   Let \(f\) : \(\Re \longrightarrow \mathcal{R}\) be given by \\
   $$    f(x) =  \begin{cases}
           \frac{e^{-x^2}}{2}  &  if \quad x > 0 \\
      (-x)^\frac{1}{3}   & if \quad x < 0   \\
       0    & if  \quad  x = 0  \\ 
       \end{cases} 
       $$
  Those days were miserable! One can refer the book of Apostol [1] or that of Rudin [4] for many more such stuff.  
 \subsection*{3.5  Matrices : the ultimate savior} 
 This was the easiest among all the math I did throughout my life . We studied 2×2 and 3×3 square matrices of the form
$$
 A = \begin{bmatrix}
    a & b \\
    c & d , 
    
    \end{bmatrix} 
    ,   B = \begin{bmatrix}
        a_{1} & a_{2} & a_{3} \\
        b_{1} & b_{2} & b_{3}  \\
        c_{1} & c_{2} & c_{3}
    \end{bmatrix}
     \\
     $$
and 2×3 rectangular matrices like \\
$$ C = \begin{bmatrix}
      a_{1} & a_{2} & a_{3} \\
      b_{1} & b_{2} & b_{3} 
  \end{bmatrix}
   \\
   $$
and even the general m×n matrix of the form \\
$$    \begin{bmatrix}
       a_{1,1} & a_{1,2} & ... &  a_{1,n} \\
       a_{2,1} & a_{2,2} & ... &  a_{2,n} \\
         .     &     .   & . & . &              \\
      .     &     .   & . & . &  \\
       .     &     .   & . & . & \\
       a_{1} & a_{2} &  ... & a_{3} \\
   \end{bmatrix}
    \\
    $$
   \section*{4  A skill that I acquired from the college:  \LaTeX typesetting}
   As part of the skill enhancement courses, in the third semester, we studied the course ‘\LaTeX
   typesetting for beginners’. We were following the book authored by Stefan Kottwitz [2].
   \subsection*{4.1 My wedding invitation}
   We got an exercise to prepare a wedding invitation card using \emph{LaTeX} and I prepared the following. 
   \begin{figure}
       \centering
       \includegraphics[width=0.7\linewidth]{figure/WhatsApp Image 2024-11-23 at 14.44.29.jpeg}
       \caption{wedding card}
       \label{fig:enter-label}
   \end{figure}
   \newpage
   \section*{Acknowledgment}
    I sincerely thank my fellow students for helping me complete this assignment.
    \begin{thebibliography}{8}
    \bibitem{} Apostol, T.M.,1991. Calculus, Volume \emph{ John Wiley and Sons.}
    \bibitem{} Kottwitz, S.,2011. \emph{LaTeX}  beginner’s guide. Packt Publishing Ltd.
    \bibitem{}  Nambudiripad, K.B.M.,2014. \emph{LaTeX}   for Beginners. Narosa Publishing House, Delhi.
    \bibitem{}   Rudin, W.,1964. Principles of mathematical analysis (Vol. 3). New York: McGraw-hill.
    \end{thebibliography}
    \cleardoublepage
   \end{document}

